\documentclass[a11paper]{article}

\usepackage{karnaugh-map}
\usepackage{subcaption}
\usepackage{tabularx}
\usepackage{titlepage}
\usepackage{document}
\usepackage{booktabs}
\usepackage{multicol}
\usepackage{float}
\usepackage{varwidth}
\usepackage{graphicx}
% \usepackage[toc,page]{appendix}
\usepackage[dvipsnames]{xcolor}

\title{Rapport d'APP}

\class{Architecture des ordinateurs}
\classnb{GIF310}

\teacher{Marc-André Tétrault}

\author{
  \addtolength{\tabcolsep}{-0.4em}
  \begin{tabular}{rcl} % Ajouter des auteurs au besoin
      Benjamin Chausse & -- & CHAB1704 \\
      Shawn Couture    & -- & COUS1912 \\
  \end{tabular}
}

\newcommand{\todo}[1]{\begin{color}{Red}\textbf{TODO:} #1\end{color}}
\newcommand{\note}[1]{\begin{color}{Orange}\textbf{NOTE:} #1\end{color}}
\newcommand{\fixme}[1]{\begin{color}{Fuchsia}\textbf{FIXME:} #1\end{color}}
\newcommand{\question}[1]{\begin{color}{ForestGreen}\textbf{QUESTION:} #1\end{color}}

\begin{document}
\pagenumbering{gobble}  % Disable page numbers
\maketitle
\newpage
\tableofcontents
\newpage
\pagenumbering{arabic}  % Start page numbering
\setcounter{page}{1}    % Start from page 1

\section{Performances d'organisations}

Une analyse des différentes organisations proposées est effectuée dans cette
section. Cette analyse donne une vision des gains de performances qu'un
changement de la méthode d'exécution des instructions amènerait.

\subsection{Analyse du code de référence}

Le code de référence contient des pseudo-instructions assembleur que MARS
transforme en instructions standard. Premièrement, les instructions "li" sont
des pseudo-instructions. Cependant, ces dernières sont traduites en une seule
instruction et donc n'ont pas d'impact sur le nombre d'instructions total
et le temps d'exécution. Malheureusement, les 4 instructions "sw" et "lw"
dans \verb|boucle_interne| sont des pseudo-instructions puisqu'elles indexent
des tableaux directement. Ceci n'est pas possible en MIPS. MARS remplace ces
pseudo-instructions par 3 instructions différentes. En premier, il charge
l'adresse du tableau dans un registre, ensuite il décale l'adresse pour
qu'elle pointe sur le bon indice du tableau, et finalement un vrai "lw" ou
"sw" est effectué à l'indice 0 de l'adresse décalée. Ceci veut dire que la
\verb|boucle_interne| a 6 instructions supplémentaires.

\subsection{Organisation unicycle}

%\todo{documenter sous forme algébrique l'organisation unicycle}
%\todo{Calculer en cycles d'horloge l'organisation unicycle}

Le calcul du temps d'exécution en cycles d'horloge pour une organisation
unicycle est aussi simple que de compter le nombre d'instructions exécutées.
Aucune bulle ou vidange n'est nécessaire.

La boucle interne a un total de $15$ instructions qui, une fois traduites,
donne un total de $23$. Ces $23$ instructions sont exécutées $4$ fois. Au 5e
appel, une seule instruction est exécutée, soit la comparaison \verb|beq|, qui
branche dans \verb|finBoucleInterne|, laquelle contient $2$ instructions. La
formule pour la boucle interne est donc:

\begin{equation}
  T = 4\times(23)+(2+1)
\end{equation}

La boucle externe est aussi exécutée $4$ fois. Elle contient $2$ instructions,
puis une exécution complète de la boucle interne. À la 5e exécution, une seule
instruction est exécutée pour sortir de la boucle et appeler
\verb|finBoucleExterne|. Cette dernière contient $2$ instructions. La formule
pour la boucle externe est donc la suivante:

\begin{equation}
T = 4\times(2+4\times(23)+(2+1)) + (2+1)
\end{equation}

Reste juste le \verb|main| à inclure, qui contient $2$ instructions. Le temps
d'exécution en cycles d'horloge est donc:

\begin{align}
  T &= 2+ 4\times(2+4\times(23)+(2+1)) + (2+1) \\
  T &= 4\times(4\times(23)+5) + 5 \\
  T &= 4\times(97) + 5 \\
  T &= 393
\end{align}

\subsection{Organisation en pipeline avec branchement au 4\textsuperscript{e} étage}

%\todo{identifier et calculer toutes les pénalités causées par deux organisations avec branchement au 4e étage en pipeline}
%\todo{identifier et calculer toutes les pénalités causées par deux organisations avec branchement au 2e étage}

L'analyse de l'organisation en pipeline avec branchement au 4\textsuperscript{e}
étage est faite sans unité d'avancement ni détection des aléas de données. Ceci
dit, l'ajout de bulles est nécessaire pour éviter que des instructions utilisant
les mêmes registres entrent en concurrence. Si ce n'est pas fait, une
instruction pourrait additionner dans un certain registre, l'instruction
suivante l'utiliserait sans avoir la bonne valeur, car l'instruction
précédente n'aurait pas terminé de s'exécuter. Après une analyse de
l'organisation, une bulle de $2$ instructions minimum est nécessaire entre
instructions concurrentes. Vu que les branchements sont au 4\textsuperscript{e}
étage, une bulle de $3$ \verb|nop| est nécessaire après chaque appel
d'instructions de format J pour éviter que les instructions suivant les
branchements soient chargées dans le pipeline alors qu'elles ne le devraient pas.

Finalement, il faut prendre en compte la traduction des pseudo-instructions.
En effet, elles sont traduites en instructions qui utilisent des registres
concurrents! L'ajout de $2$ bulles de $2$ instructions par pseudo-instruction
est donc nécessaire : la première entre le chargement de l'adresse du tableau
et son déplacement, la deuxième entre le déplacement de l'adresse et
l'écriture ou la lecture à cet endroit. Cependant, les bulles ajoutées avant
l'appel des pseudo-instructions ne sont pas nécessaires puisque le registre
n'est utilisé que $2$ instructions plus loin. Ceci réduit de $8$ \verb|nop|
le total de la boucle interne.

Après analyse du programme de référence, le \verb|main| devient $4$
instructions, la boucle interne a un nouveau total de $55$ instructions, la
boucle externe en a $8$ au lieu de $2$, le \verb|main| en a $4$ au lieu de $2$,
la fin de boucle externe en a $8$ au lieu de $2$, et la fin de boucle interne
en a $5$ au lieu de $2$. La nouvelle formule est donc :

\begin{align}
  T &= 4+ 4\times(7+4\times(55)+(5+1)) + (8+1) \\
  T &= 945
\end{align}

\subsection{Organisation en pipeline avec branchement au 2\textsuperscript{e} étage}

Cette organisation change l'emplacement des branchements et les monte de
$2$ étages. Ils sont donc effectués plus tôt dans la chaîne d'événements du
pipeline. Cela a pour effet que $2$ des $3$ \verb|nop| nécessaires pour les
instructions de format J peuvent être enlevés. Par la suite, cette
organisation, pour notre cas d'étude, implémente une unité d'avancement. Ceci
permet aux résultats des instructions d'être disponibles plus tôt et de ne pas
devoir attendre leur mise en mémoire (pour les instructions qui ne sont pas
des branchements), éliminant les bulles entre les accès des instructions
concurrentes. Cependant, les branchements sont exécutés plus haut que
l'emplacement de retour de l'unité d'avancement. Cela veut dire que la bulle
reste nécessaire lorsque les instructions concurrentes sont avec des
instructions de branchement. De plus, une unité de détection des aléas est
présente. Cette dernière est plutôt complexe, mais enlève complètement la
nécessité de rajouter des \verb|nop| manuellement pour créer des bulles.
Malheureusement, on doit tout de même calculer les délais ajoutés, tel que
mentionné plus haut. Le \verb|syscall| a encore besoin de ses $4$
instructions précédentes et d'une bulle de $2$ pour le registre \verb|$v0|
avant son appel. De même, les pseudo-instructions n'ont plus de bulles et ont
donc le même impact d'ajout d'instructions que dans l'organisation unicycle.

Cela veut dire que le \verb|main| a encore $4$ instructions, la boucle
interne en a maintenant $25$, la boucle externe en a maintenant $5$, la fin
de boucle externe en a encore $8$ et la fin de boucle interne en a maintenant
$3$. La formule algébrique est donc la suivante:


\begin{align}
  T &= 4+ 4\times(5+4\times(25)+(3+1)) + (8+1) \\
  T &= 449
\end{align}

\subsection{Temps d'exécution}

%\todo{calculez le temps d'exécution en vous basant sur une vitesse d'opération de 25 ns pour l'organisation unicycle}
%\todo{calculez le temps d'exécution en vous basant sur une vitesse d'opération de 10 ns pour l'organisation pipeline}

Pour notre cas d'étude, l'organisation unicycle a un temps de $25\,\text{ns}$
par instruction, tandis que les organisations en pipeline prennent $10\,\text{ns}$.
La formule triviale suivante indique que le temps d'exécution est le total des
instructions multiplié par le temps d'un cycle d'horloge.

\begin{equation}
\text{T}_{\text{sys}} = \text{instructions} \times \text{T}_{\text{CPU}}
\end{equation}

L'unicycle prend donc $9825$ nanosecondes, le pipeline avec branchement au
4\textsuperscript{e} étage prend $9450$ nanosecondes, et l'organisation en
pipeline avec branchement au 2\textsuperscript{e} étage prend $4490$ nanosecondes.

\subsection{Analyse des résultats}

L'organisation en pipeline avec branchement au 2\textsuperscript{e} étage est
$2{,}2\times$ plus performante que l'organisation unicycle, avec l'avantage
de n'avoir aucune différence dans les codes assembleurs. Cependant, la
modification de l'organisation vers une méthode pipeline peut être coûteuse en
développement et nécessite un changement matériel. Quant à l'organisation avec
branchement au 4\textsuperscript{e} étage, le gain de performance n'est que de
$0{,}03\times$, ce qui est complètement négligeable, et nécessite l'ajout
manuel de bulles dans le langage assembleur. Les gains ne sont pas à négliger
cependant, puisqu'une amélioration de $2{,}2\times$ atteint presque la demande
d'amélioration de $3\times$, sans nécessairement faire $3\times$ moins
d'instructions, tel que demandé.

\section{Performances SIMD}

\subsection{Instructions identifiées}

\subsubsection{Charger un vecteur}
Vu que le programme doit utiliser des vecteurs, leur chargement doit être
possible. Donc, une instruction SIMD \verb|lwv| est à implémenter. Elle vient
avec le bonus d'enlever la traduction des pseudo-instructions.

\subsubsection{Multiplication}
Le code de référence effectue une multiplication entre deux données de deux
vecteurs dans la boucle interne. Or, ceci peut être fait dans une instruction
SIMD, ce qui éliminera éventuellement la boucle interne. L'instruction
\verb|multv| permet de multiplier deux vecteurs et de mettre le résultat dans
un autre.

\subsubsection{Somme d'un vecteur}
L'analyse du code de référence montre qu'une somme des données est effectuée.
Or, l'utilisation d'une instruction SIMD est possible ici. Une instruction
spéciale appelée \verb|sumv| permet d'additionner les $4$ valeurs d'un vecteur
ensemble et de mettre le résultat dans un registre standard. Ceci permet de
mettre immédiatement le résultat au bon endroit dans le vecteur de sortie.

\subsection{Nouveau temps d'exécution}
Dans le nouveau code de référence, le \verb|main| a $7$ instructions. La boucle
interne a complètement disparu grâce aux instructions SIMD. La boucle
externe a $9$ instructions. La comparaison est à la toute fin, ce qui réduit
d'une instruction car la boucle n'est pas exécutée $5$ fois, mais seulement
$4$. La fin de boucle est de $2$ instructions. De plus, il n'y a plus de
pseudo-instructions.

Ceci dit, la formule de calcul du nombre d'instructions est maintenant la
suivante :

\begin{align}
  T &= 7 + 4\times(9) + 2 \\
  T &= 45
\end{align}

\subsection{Comparaison avec l'unicycle}

Le total d'instructions passe de $393$ à juste $45$. C'est un gain de $8{,}7$ !
Ce gain démontre amplement qu'il est possible de réduire d'au minimum $3$ fois
le nombre d'instructions total pour la problématique, étant donné la nature
similaire des deux programmes assembleurs.

%\todo{identifiez les instructions qui seraient à convertir en SIMD}

%\todo{calculez le nouveau temps d'exécution en cycles d'horloge, pour enfin le
%comparer avec celui en unicycle}

%\todo{dire si le gain de performance permet d'espérer d'atteindre les objectifs
%de la problématique}



\section{Performances des mémoires sur processeur unicycle}

\subsection{Pour la mémoire de données}

\subsubsection{Cas de la DRAM de la problématique}

La DRAM de base a une pénalité de $10$ coups d'horloge par accès. Chaque accès
fournit 2 mots séquentiels. De base, aucune cache de données n’est incluse.
Donc, chaque accès aux données nécessite $10$ coups d'horloge. Le mode de
lecture "2-word burst" n’apporte aucun avantage puisque les données ne sont pas
sauvegardées dans une cache. Cela veut donc dire qu’un des deux mots chargés
n’est pas utilisé et qu’un autre accès à la DRAM est nécessaire plus tard pour
aller chercher le deuxième mot ! Cependant, si des instructions SIMD sont
utilisées, alors ce mode de lecture permet de gagner du temps lors du
chargement de vecteurs. Charger un vecteur de $4$ prendrait uniquement $20$
coups d’horloge plutôt que $40$ sans ajout d’instructions SIMD.

Le calcul du temps d’accès pour l’implémentation de mémoire sans cache est très
simple. Chaque appel de \verb|sw| et \verb|lw| cause des délais d’accès.
Les instructions immédiates n’en causent aucun puisque les données sont
directement dans l’instruction. Une fois que les données sont dans les registres,
le coût d’accès est nul.

Le programme matriciel de référence fait $4$ appels à ce genre
d’instructions à l’intérieur de la boucle interne. La formule mathématique
des coups d’horloge supplémentaires est donc simplement :

\begin{align}
  \text{T} &= 10\times4\times(4\times4) \\
  \text{T} &= 10\times4\times(4\times4) \\
  T        &= 640
\end{align}

Si on rajoute les instructions de l'organisation unicycle, on obtient un
total de $1033$ coups d'horloge, sans compter les pénalités de mémoire
d'instructions !

\subsubsection{Cas avec une cache attachée}

Pour réduire le nombre de pénalités engendrées par l'accès à la DRAM, il
suffit d'ajouter une cache. Pour ce cas d'étude, il s'agit d'une cache de $256$ blocs
avec $2$ mots de $32$ bits par bloc, avec un mode de DRAM "write-through".
Dans les caches, les premiers accès causent une pénalité de lecture. Les accès
suivants, aux mêmes données, ne causent pas de pénalité puisque le programme n'a
plus besoin d'accéder à la DRAM. Hors, les vecteurs d'entrée et de sortie sont
de $4$ mots. Sachant que la cache enregistre des données contiguës à l'écriture,
la lecture initiale des vecteurs cause $2$ pénalités d'accès.
Pour la matrice, $8$ pénalités ont lieu pour toute la mettre dans la cache par
groupe de $2$ mots. La cache ayant $256$ blocs, aucun remplacement de blocs
aura lieu.

Le calcul reste simple, uniquement les pénalités d'accès mentionnées plus haut
doivent être calculées puisque n'importe quelle autre lecture ou écriture
n'a plus besoin d'accéder à la DRAM. Voici donc la formule :

\begin{align}
  \text{T} &= 10\times(2+2+8) \\
  \text{T} &= 120
\end{align}

\subsection{Pour les instructions}
\subsubsection{Cas de la DRAM de la problématique}

Le calcul des délais d'accès en mémoire pour les instructions est
relativement facile pour le cas de la DRAM de la problématique. En effet,
sans cache, chaque appel d'instruction engendre un délai d'accès ! Même si
la DRAM fournit $2$ mots (donc deux instructions) par appel, l'organisation
unicycle ne peut en exécuter qu'une et sans cache, elle ne peut pas
enregistrer la prochaine instruction. Donc chaque instruction crée $10$
coups d'horloge supplémentaires.

Le calcul est relativement simple, il s'agit de multiplier le nombre total
d'instructions calculé plus haut par $10$ pour obtenir les coups d'horloge
supplémentaires engendrés par leur accès. On obtient donc la performance
suivante :

\begin{equation}
  \text{T}=393\times10=3930
\end{equation}

\subsubsection{Cas avec une cache attachée}

L'attachement d'une cache à la DRAM permet l'enregistrement des instructions
pour éviter d'aller les chercher avec pénalité à chaque accès. De plus, les
instructions dans les boucles et la fin de boucle interne sont chargées une
seule fois dans la cache. Heureusement, la cache de $256$ blocs de $2$ mots
proposée par le cas d'étude a assez d'espace pour qu'aucun remplacement de
bloc n'ait lieu.

Le calcul reste simple : une pénalité a lieu tous les deux accès (par bloc).
Il suffit donc de compter le nombre d'instructions uniques dans le programme,
de le diviser par $2$, puis de multiplier par $10$ ns pour obtenir la pénalité
totale en nanosecondes liée à l'accès à la DRAM.

Dans notre cas, on compte $23$ instructions uniques, incluant les traductions
des pseudo-instructions. La formule mathématique pour le nombre de cycles
supplémentaires est donc la suivante :

\begin{align}
  \text{T} &= 10\times\lceil23/2\rceil \\
  \text{T} &= 10\times12 \\
  \text{T} &= 120
\end{align}

\subsection{Analyse des résultats}

Après l'analyse des configurations de cache, il est évident que l'ajout d'une
mémoire cache à la DRAM apporte des gains significatifs en performance.
En effet, un gain d'un facteur $32$ est observé pour la mémoire
d'instructions, et un gain de $5{,}3$ pour les accès aux données.
L'ajout d'une cache améliore donc nettement la performance, même si
cela implique un changement matériel coûteux et ne réduit pas
le nombre d'instructions exécutées. Cependant, un gain de performance
de $32$ pourrait significativement réduire leur problème.

\section{Configuration des caches}

Le programme de référence ainsi que l'algorithme de Viterbi utilisent des
tableaux séparables en vecteurs de taille $4$. En effet, la meilleure
implémentation SIMD pour ces programmes utilise des instructions gérant
des vecteurs de $4$ éléments. De plus, les deux algorithmes exploitent la
même structure de données : une matrice $4\times4$ et deux tableaux de
taille $1\times4$. Optimiser pour l'un optimise donc partiellement l'autre.

La cache gère les requêtes et est très versatile. Par exemple, pour une
cache avec blocs de $4$ mots et une DRAM avec blocs de $2$ mots,
la cache récupérera ses $4$ mots via deux appels DRAM. Si une instruction
demande $8$ mots, la cache les fournira en un temps similaire à celui
d'une instruction demandant un seul mot.

Cela simplifie grandement l'optimisation de la cache. Le nombre optimal
de mots par bloc est $2$, ce qui permet des remplacements partiels.
Ainsi, si un vecteur de $4$ mots est mis en cache, il occupe deux blocs.
Un remplacement partiel n'entraîne alors que la récupération partielle
des données, réduisant la pénalité DRAM. Avec des blocs de $4$ mots,
le vecteur complet serait remplacé, nécessitant deux pénalités DRAM.

Concernant le type de cache (directe, associatif par ensemble, ou totalement
associative), une cache associative par ensemble paraît pertinente.
Les vecteurs réutilisés peuvent être placés dans un premier ensemble,
et les données des matrices dans un second, car l'accès aux matrices
$4\times4$ se fait une seule fois sans réutilisation. Cela réduit la taille
de la cache mais augmente la complexité. La cache se remplit de façon
automatique, donc le code assembleur doit garantir un ordre d'accès
approprié pour que les données des matrices ne se retrouvent pas dans
le même ensemble que les vecteurs.

Dans ces conditions, une cache associative par ensemble avec
$8$ blocs de $2$ mots par ensemble et $2$ ensembles semble
très optimisée pour le calcul de référence, en supposant un ordre
d'accès aux données optimisé. Ce serait préférable à une cache
directe de $32$ blocs de $2$ mots.

L'analyse du code de référence confirme que les vecteurs d'entrée
et de sortie sont chargés en premier, donc seront mis dans
le premier ensemble, tandis que le deuxième sera rempli avec
les données de la matrice. Cette organisation optimise à la fois
les accès ratés et la taille de la cache.

\subsection{Optimisation spécifique à Viterbi}

Viterbi utilise un index de boucle stocké en mémoire pour éviter
les conflits avec les appels de fonctions internes. L'optimisation
ci-dessus ne prend pas en compte l'ordre d'accès du code Viterbi,
ni le nombre de données différentes utilisées.

\section{Intégration}

\subsection{Ébauche de priorisation en fonction du temps ressenti}

Les modifications logicielles sont les moins chronophages. Si un non-respect total
des standards MIPS optimise significativement le code Viterbi ou le code de
référence, cela devrait être prioritaire.

Viennent ensuite les modifications matérielles. L'entreprise dispose déjà d'une
base permettant de créer des caches modulables. L'ajout d'une cache d'instructions
est prioritaire, car elle apporte un gain de performance important pour les cas
d'étude, suivi par la cache de données.

Le changement d’organisation en pipeline avec branchement au 4e étage est moins
complexe que celui au 2e étage, qui requiert des modules logiques supplémentaires.
Chacune apporte un gain de performance. Des architectures similaires existent en
open source, notamment pour des processeurs simples comme ceux étudiés. Toutefois,
cela nécessite une restructuration matérielle, donc un coût potentiellement élevé.

Enfin, l’implémentation des instructions SIMD est la plus coûteuse. Elle exige
non seulement des modifications matérielles majeures, mais aussi un temps de
développement important et risqué. Certaines instructions peuvent n’être utiles
que pour ce cas précis, ce qui réduit la versatilité.

\subsection{Ébauches des gains en ordre de priorisation}
Sachant que le total de coups d'horloge sans aucun changement, pour
l'organisation unicycle est le suivant:
\begin{align}
  \text{T} &= 10 \times \text{instructions} + 10 \times \text{accès données} + \text{instructions} \\
  &= 10 \times 393 + 10 \times 64 + 393 \\
  &= 3930 + 640 + 393 \\
  &= 4963
\end{align}

Le plus gros gain vient de l’implémentation des instructions SIMD, qui réduit
le nombre d’instructions de $393$ à $45$. Ce changement diminue aussi bien les
pénalités d’accès à la cache d’instructions que celles à la DRAM données.
En effet, dans le nouveau programme, au lieu de $4$ accès DRAM par boucle exécutée
$4$ fois, seuls $2$ accès impactent la mémoire données, exécutés $4$ fois.
La nouvelle formule devient :

\begin{align}
  \text{T} &= 10 \times 45 + 10 \times (2 \times 4 + 1) + 45 \\
  &= 450 + 90 + 45 \\
  &= 585
\end{align}

Soit une amélioration d’un facteur $\approx 8.5$.

L’ajout d’une cache d’instructions apporte initialement un gain de facteur $32$
sur les pénalités d’accès. Appliqué au calcul initial, ce gain effectif est
d’environ $4.3$, ce qui dépasse les gains liés à l’organisation pipeline ou
la cache de données. Ainsi, le nombre de pénalités d’accès passe de $45$ à $18$.
Appliqué au code SIMD, le total devient :

\begin{align}
  \text{T} &= 10\times(18/2) + 10\times(2\times4+1) + 36 \\
  \text{T} &= 90 + 90 + 36 \\
  \text{T} &= 216
\end{align}
Ce qui est un gain de $2.6$ après l'implémentation du SIMD. Avant
l'implémentation du SIMD, c'est un gain de $4.30$.

Changé l'organisation pour être en pipeline 2e étages serais la prochaine chose
à faire. En effet, la durée d'un coup d'horloge baisse de \SI{25}{\nano\s} à \SI{10}{\nano\s}, mais
le nombre d'instructions total executé change. le gain observé reste à peu prêt
$2.2$. Ce calcul est fastidieux à refaire pour le code de référence SIMD et le
gain total ne changera pas de beaucoup du $2.2$ initialement calculé pour
l'organisation de base.

Pour l'ajout de la cache de données, le gain final n'est pas de beaucoup. En
effet, avec le code de référence initiale, un gain de $1.17$ est observé. Si
cette cache est implémenté après l'implémentation du SIMD, le gain est
d'uniquement $1.01$ car le nombre de coup d'horloge passe de $565$ à $555$.
S'il est implémenté après l'implémentation de la cache d'instructions et le
SIMD, alors le gain est un peu plus. On passe de $216$ à $186$ pour un gain
de $1.16$.

\subsection{Système proposé}
Le système proposé est une organisation avec branchement aux 2e étages incluant
une unité d'avancement et de détection d'hazard, avec des caches

%\todo{proposer un système optimal et prioriser les modifications par leurs
%gains effectifs sur la performance et leur temps de développement}

\end{document}
