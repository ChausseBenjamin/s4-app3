\documentclass[a11paper]{article}

\usepackage{karnaugh-map}
\usepackage{subcaption}
\usepackage{tabularx}
\usepackage{titlepage}
\usepackage{document}
\usepackage{booktabs}
\usepackage{multicol}
\usepackage{float}
\usepackage{varwidth}
\usepackage{graphicx}
% \usepackage[toc,page]{appendix}
\usepackage[usenames,dvipsnames]{xcolor}

\title{Rapport d'APP}

\class{Logique Combinatoire}
\classnb{GEN420 \& GEN430}

\teacher{Marwan Besrour \& Gabriel Bélanger}

\author{
  \addtolength{\tabcolsep}{-0.4em}
  \begin{tabular}{rcl} % Ajouter des auteurs au besoin
      Benjamin Chausse & -- & CHAB1704 \\
      Shawn Couture    & -- & COUS1912 \\
  \end{tabular}
}

\newcommand{\todo}[1]{\begin{color}{Red}\textbf{TODO:} #1\end{color}}
\newcommand{\note}[1]{\begin{color}{Orange}\textbf{NOTE:} #1\end{color}}
\newcommand{\fixme}[1]{\begin{color}{Fuchsia}\textbf{FIXME:} #1\end{color}}
\newcommand{\question}[1]{\begin{color}{ForestGreen}\textbf{QUESTION:} #1\end{color}}

\begin{document}
\maketitle
\newpage
\tableofcontents
\newpage

\section{Performances d'organisations}
Une analyze des différentes organisations proposées est effectuée dans cette section. Cette analyze donne une vision des gains de performances qu'un changement de la méthode
d'exécution des instructions amènerait.

\subsection{Analyze du code de référence}
Le code de référence contient des pseudo-instructions assembleurs que MARS transforme en instructions standard. Premièrement, les instructions "li" sont des pseudo-instructions.
Cependant, ces dernières sont traduite en une seule instruction et donc n'ont pas d'impacte sur le nombre d'instructions totaux et le temps d'exécution.
Malheureusement, les 4 instructions "sw" et "lw" dans \verb|boucle_interne| sont des pseudo-instructions puisqu'ils index sur des tableaux directement. Ceci n'est pas possible en
MIPS. MARS remplace ces pseudo-instructions par 3 différentes instructions. En premier il charge l'addresse du tableau dans un registre, ensuite il décale l'addresse pour qu'elle
pointe sur la bonne index du tableau et finalement un vrai "lw" ou "sw" est effectué à l'index 0 de l'addresse décalé. Ceci veut dire que la \verb|boucle_interne| a 6 instructions
supplémentaire.

\subsection{Organisation unicycle}
%\todo{documenter sous forme algébrique l'organisation unicycle}
%\todo{Calculer en cycles d'horloge l'organisation unicycle}
Le calcul du temps d'exécution en cycle d'horloge pour une organisation unicycle est aussi simple que de compter le nombre d'instructions exécuté. Aucune bulles ou vidanges est 
nécessaire. 

La boucle interne a un total de $15$ instructions qui une fois traduite donne un total de $23$. Ces 23 instructions sont exécuté $4$ fois. Au 5e appel, une seule instruction est
exécuté, soit la comparaison \verb|beq|, qui branche dans \verb|finBoucleInterne| qui est $2$ instructions. La formule pour la boucle interne est donc:
$$
T = 4\times(23)+(2+1)
$$

la boucle externe est aussi exécuté $4$ fois. Elle contient $2$ instructions puis une exécution complète de la boucle interne. À la 5e exécution, une seule instruction est exécuté
pour sortie de la boucle et appeller \verb|finBoucleExterne|. Cette dernière est $2$ instructions. La formule pour la boucle externe est donc la suivante:
$$
T = 4\times(2+4\times(23)+(2+1)) + (2+1)
$$
Reste juste le \verb|main| à inclure, qui est $2$ instructions. Le temps d'exécution en coup d'horloge est donc:
$$
T = 2+ 4\times(2+4\times(23)+(2+1)) + (2+1)
$$
$$
T = 4\times(4\times(23)+5) + 5
$$
$$
T = 4\times(97) + 5
$$
$$
T = 393
$$

\subsection{Organisation en pipeline avec branchement au 4e étage}
%\todo{identifier et calcuer toutes les pénalités causées par deux organisation avec branchement au 4e étage en pipeline}
%\todo{identifier et calcuer toutes les pénalités causées par deux organisation avec branchement au 2e étage}
L'analyze de l'organisation en pipeline avec branchement au 4e étage est fait sans unité d'avancement ni de détection des aléas de données. Ceci dit, l'ajout de bulles est
nécessaire pour évité que des instructions utilisant les même registres rentre en concurance. Si ce n'est pas fait, une instructions pourrait additionner dans un certain registre, 
l'instruction suivante l'utiliserait mais n'aurait pas la bonne valeur car l'instruction précédente n'a pas terminé d'exécuté. Après une analyze de l'organisation, une bulle de $2$
instructions minimum est nécessaire entre instructions concurente. Vu que les branchements sont au 4e étage, une bulle de $3$ \verb|nop| est nécessaire après chaque appel
d'instructions de format J pour évité que les instructions suivant les branchements soit charger dans le pipeline lorsqu'il ne devrait pas.

Finalement, il faut prendre en compte la traduction des pseudo-instructions. En effet, ils sont traduits en instructions qui utilise des registres concurents! l'ajout de 2 bulles
de $2$ instructions par pseudo-instructions est donc nécessaire. La première entre le chargement de l'addresse du tableau et son déplacement, la deuxième entre le déplacement de
l'addresse et l'écriture ou la lecture à cet endroit. Cependant, les bulles ajouter avant l'appel des pseudo-instructions ne sont donc pas nécessaire puisque le registre n'est
qu'utilisé que 2 instructions plus loin. Ceci réduit de 8 \verb|nop| le total de la boucle interne.

Après analyze du program de référence, le \verb|main| devient 4 instructions, la boucle interne à un nouveau total de $55$ instructions, la boucle externe en a $8$ au lieu de $2$,
le main en a $4$ au lieu de $2$, la fin de boucle externe en a $8$ au lieu de $2$ et la fin de boucle interne en a $5$ au lieu de $2$.
La nouvelle formule est donc:
$$
T = 4+ 4\times(7+4\times(55)+(5+1)) + (8+1)
$$
$$
T = 945
$$

\subsection{Organisation en pipeline avec branchement au 2e étage}
Cette organisation change l'emplacement des branchements et les montent de $2$ étages. Ils sont donc effectué plus tôt dans la chaine d'évènement des pipelines. Ça à pour effet que
$2$ des $3$ \verb|nop| nécessaire pour les instructions de format J peuvent être enlevé. Par la suite, cette organisation, pour notre cas d'étude, implémente une unité d'avancement.
Ceci permet aux résultats des instructions d'être disponible plus tôt et de ne pas devoir attendre leurs mise en mémoire (pour les instructions qui ne sont pas des branchements), 
éliminant les bulles entre les accès des instructions concurentes. Cependant, les banchements sont exécuté plus haut que l'emplacement de retour de l'unité d'avancement. Ça veut dire
que la bulle reste nécessaire lorsque les instructions concurentes sont avec des instructions de branchements!
De plus, une unité de détection des aléas est présente. Cette dernière est plutôt complexe mais enlève 
complètement le nécessaire de rajouter des \verb|nop| manuellement pour créé des bulles. Malheureusement, on doit tout de même calculer les délais ajoutés tel que mentionné plus haut.
Le \verb|syscall| a encore besoion de ces $4$ instructions le précédent et d'une bulle de $2$ pour le registre \verb|$v0| avant son appel. De même, les pseudo-instructions n'ont plus
de bulles et on donc le même impacte d'ajout d'instruction que l'organisation unicycle.

Ceci veut dire que, le \verb|main| à encore $4$ instructions, la boucle interne en a maintenant $25$, la boucle externe en a maintenant $5$, la fin de boucle externe en a encore $8$
et la fin de boucle interne en a maintenant $3$. La formule algébrique est donc la suivante:
$$
T = 4+ 4\times(5+4\times(25)+(3+1)) + (8+1)
$$
$$
T = 449
$$

\subsection{Temps d'exécution}
%\todo{calculez le temps d'exécution en vous basant sur une vitesses d'opération de 25 ns pour l'organisation unicycle}
%\todo{calculez le temps d'exécution en vous basant sur une vitesses d'opération de 10 ns pour l'organisation pipeline}
Pour notre cas d'étude, l'organisation unicycle à un temps de $25ns$ par instructions tandis que les organisations en pipeline en prennent $10ns$.
La formule trivial suivante indique que le temps d'exécution est le total des instructions multiplié par le temps d'un coup d'horloge.
$$
\text{T}_{\text{sys}} = \text{instructions}\times\text{T}_{\text{CPU}}
$$
L'unicycle prend donc $9825$ nanosecondes, le pipeline avec branchement au 4e étage prend $9450$ nanosecondes et l'organisation en pipeline
avec branchement au 2e étage prent $4490$ nanosecondes.

\subsection{Analyze des résultats}
L'organisation en pipeline avec branchement au 2e étage est $2.2\times$ plus performant que l'organisation unicycle avec l'avantage de d'avoir aucune différence dans
les codes assembleurs. Cependant, la modification de l'organisation pour une méthode pipeline peut être couteuse en developpement et nécessite un changement 
matérielle. Pour ce qui est de l'organisation avec branchement au 4e étage, le gain de performance n'est que de $0.03$, complètement négligable et nécessite l'ajout
manuelle de bulles dans le langage assembleur.
Les gains ne sont pas à négligé cependant puisqu'une amélioration de $2.2\times$ atteint presque la demande d'amélioration de $3\times$ sans nécessairement faire 
$3\times$ moins d'instructions tel que demandé.


\section{Performances SIMD}

\todo{identifiez les instructions qui seraient à convertir en SIMD}

\todo{calculez le nouveau temps d'exécution en cycles d'horloge, pour enfin le
comparer avec celui en unicycle}

\todo{dire si le gain de performance permet d'espérer d'atteindre les objectifs
de la problématique}



\section{Performances des mémoires sur processeur unicycle}
\subsection{Pour la mémoire de données}
\subsubsection{Cas de la DRAM de la problématique}
La DRAM de base à une pénalité de 10 coups d'horloge par accès. Chaque accès fournis 2 mots séquentielles. De base, aucune cache de donnée est inclus. Donc chaque accès aux données
nécessite 10 coups d'horloge. Le mode de lecture "2-word burst" donne aucun avantage puisque les données ne sont pas sauvegarder dans une cache. Ça veut donc dire que un des deux
mots chargé n'est pas utilisé et un autre accès à la DRAM est nécessaire plutard pour allé chercher le deuxième mot! Cependant, si des instructions SIMD sont utilisés alors ce mode
de lecture vient sauver du temps lors de chargement de vecteurs. Charger Un vecteur de $4$ prendrait uniquement $20$ coup de clock plutôt que $40$ sans ajout d'instructions SIMD.

Le calcul du temps d'accès pour cette implémentation de mémoire est très simple. Chaque appel de \verb|sw| et \verb|lw| causent les délais d'accès. Les instructions immédiate en
cause aucun puisque les données sont dans l'instruction directe. Une fois que les données sont dans les registres, le coup d'accès est nulle.

Le programme matricielle de référence fait $4$ appel à ce genre d'instructions à l'intérieur de la boucle interne. La formule mathématique des pénalité est donc simplement:
$$
\text{T} = 10\times4\times(4\times4)
$$
$$
\text{T} = 10\times4\times(4\times4)
$$
$$
T=640
$$
Si on rajoute les instructions de l'organisation unicycle, on obtient un total de $1033$ coup d'horloge, sans compté les pénalité de mémoire d'instructions!

\subsubsection{Cas avec une cache attaché}
Afin de réduire le nombre de pénalité engendré par l'accès à la DRAM, l'ajour d'une cache de données est pertinente. Pour ce cas d'étude, il s'agit d'une cache de $256$ blocs avec
$2$ mots de $32$ bits par block, avec un mode de DRAM "write-through". Dans les caches, les premières accès causent une pénalité de lecture. Les accès suivantes, au mêmes données,
ne causent pas de pénalité puisque le programme n'a plus besoin d'accèder a la DRAM. Hors, les vecteurs d'entrées et de sorties sont $4$ mots. Sachant que la cache enregistres des 
données contingu à l'écriture, la lecture initiale des vecteurs causent $2$ pénalités d'accès à la DRAM par vecteur. Pour la matrice, $8$ pénalité auront lieu pour toute la mettre
dans la cache par groupe de $2$ mots. La cache ayant $256$ blocs, aucun remplacement de blocs auront lieu.

Le calcul reste simple, uniquement les pénalité d'accès mentionné plus haut ont besoin d'être calculer puisque n'importe quel autre lecture ou écriture n'a plus besoin d'accèder à
la DRAM. Voici donc la formule:
$$
\text{T} = 10\times(2+2+8)
$$
$$
\text{T} = 120
$$

\subsection{Pour les instructions}
\subsubsection{Cas de la DRAM de la problématique}
Le calcul pour les délais d'accès en mémoire pour les instructions est relativement facile pour le cas de la DRAM de la problématique. En effet, sans cache, chaque
appelle d'instruction engendre un délai d'acces! Même si la DRAM fournis $2$ mots (donc deux instructions) par appel, l'organisation unicycle ne peut qu'en exécuter
qu'une et sans cache, elle ne peut pas enregistrer la prochaine instructions. Donc chaque instructions vient créé 10 coups d'horloge de délai.

Le calcul est relativement simple, il s'agit de multiplier le nombre total d'instructions calculé plus haut à $10$ns introduit par les délais de la DRAM.
On obtient donc la performance suivante:
$$
\text{T}=393\times10=3930ns
$$

\subsubsection{Cas avec une cache attaché}
l'attachement d'une cache à la DRAM permet l'enregistrements d'instructions pour ne pas avoir besoin d'aller les chercher avec des pénalités à chaque fois.
De plus, les instructions dans les boucles et la fin de boucle interne vont être chargé dans la cache qu'une seule fois. L'analyze du code à soulevé les remarques
suivantes:


\section{Configuration des caches}

\section{Intégration}


\end{document}
