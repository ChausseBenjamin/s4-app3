\documentclass[a11paper]{article}

\usepackage{karnaugh-map}
\usepackage{subcaption}
\usepackage{tabularx}
\usepackage{titlepage}
\usepackage{document}
\usepackage{booktabs}
\usepackage{multicol}
\usepackage{float}
\usepackage{varwidth}
\usepackage{graphicx}
% \usepackage[toc,page]{appendix}
\usepackage[usenames,dvipsnames]{xcolor}

\title{Rapport d'APP}

\class{Logique Combinatoire}
\classnb{GEN420 \& GEN430}

\teacher{Marwan Besrour \& Gabriel Bélanger}

\author{
  \addtolength{\tabcolsep}{-0.4em}
  \begin{tabular}{rcl} % Ajouter des auteurs au besoin
      Benjamin Chausse & -- & CHAB1704 \\
      Shawn Couture    & -- & COUS1912 \\
  \end{tabular}
}

\newcommand{\todo}[1]{\begin{color}{Red}\textbf{TODO:} #1\end{color}}
\newcommand{\note}[1]{\begin{color}{Orange}\textbf{NOTE:} #1\end{color}}
\newcommand{\fixme}[1]{\begin{color}{Fuchsia}\textbf{FIXME:} #1\end{color}}
\newcommand{\question}[1]{\begin{color}{ForestGreen}\textbf{QUESTION:} #1\end{color}}

\begin{document}
\maketitle
\newpage
\tableofcontents
\newpage

\section{Performances d'organisations}

\subsection{Organisation unicycle}

\todo{documenter sous forme algébrique l'organisation unicycle}

\todo{Calculer en cycles d'horloge l'organisation unicycle}

\subsection{Organisation en pipeline}

\todo{identifier et calcuer toutes les pénalités causées par deux organisation
 avec branchement au 4e étage
en pipeline}

\todo{identifier et calcuer toutes les pénalités causées par deux organisation
avec branchement au 2e étage}

\subsection{Temps d'exécution}

\todo{calculez le temps d'exécution en vous basant sur une vitesses
d'opération de 25 ns pour l'organisation unicycle}

\todo{calculez le temps d'exécution en vous basant sur une vitesses
d'opération de 10 ns pour l'organisation pipeline}

\section{Performances SIMD}

\todo{identifiez les instructions qui seraient à convertir en SIMD}

\todo{calculez le nouveau temps d'exécution en cycles d'horloge, pour enfin le
comparer avec celui en unicycle}

\todo{dire si le gain de performance permet d'espérer d'atteindre les objectifs
de la problématique}

\section{Performances des mémoires sur processeur unicycle}

\section{Configuration des caches}

\section{Intégration}


\end{document}
