\documentclass[a11paper]{article}

\usepackage{karnaugh-map}
\usepackage{subcaption}
\usepackage{tabularx}
\usepackage{titlepage}
\usepackage{document}
\usepackage{booktabs}
\usepackage{multicol}
\usepackage{float}
\usepackage{varwidth}
\usepackage{graphicx}
% \usepackage[toc,page]{appendix}
\usepackage[usenames,dvipsnames]{xcolor}

\title{Rapport d'APP}

\class{Logique Combinatoire}
\classnb{GEN420 \& GEN430}

\teacher{Marwan Besrour \& Gabriel Bélanger}

\author{
  \addtolength{\tabcolsep}{-0.4em}
  \begin{tabular}{rcl} % Ajouter des auteurs au besoin
      Benjamin Chausse & -- & CHAB1704 \\
      Shawn Couture    & -- & COUS1912 \\
  \end{tabular}
}

\newcommand{\todo}[1]{\begin{color}{Red}\textbf{TODO:} #1\end{color}}
\newcommand{\note}[1]{\begin{color}{Orange}\textbf{NOTE:} #1\end{color}}
\newcommand{\fixme}[1]{\begin{color}{Fuchsia}\textbf{FIXME:} #1\end{color}}
\newcommand{\question}[1]{\begin{color}{ForestGreen}\textbf{QUESTION:} #1\end{color}}

\begin{document}
\maketitle
\newpage
\tableofcontents
\newpage

\section{Performances d'organisations}
Une analyze des différentes organisations proposées est effectuée dans cette section. Cette analyze donne une vision des gains de performances qu'un changement de la méthode
d'exécution des instructions amènerait.

\subsection{Analyze du code de référence}
Le code de référence contient des pseudo-instructions assembleurs que MARS transforme en instructions standard. Premièrement, les instructions "li" sont des pseudo-instructions.
Cependant, ces dernières sont traduite en une seule instruction et donc n'ont pas d'impacte sur le nombre d'instructions totaux et le temps d'exécution.
Malheureusement, les 4 instructions "sw" et "lw" dans \verb|boucle_interne| sont des pseudo-instructions puisqu'ils index sur des tableaux directement. Ceci n'est pas possible en
MIPS. MARS remplace ces pseudo-instructions par 3 différentes instructions. En premier il charge l'addresse du tableau dans un registre, ensuite il décale l'addresse pour qu'elle
pointe sur la bonne index du tableau et finalement un vrai "lw" ou "sw" est effectué à l'index 0 de l'addresse décalé. Ceci veut dire que la \verb|boucle_interne| a 6 instructions
supplémentaire.

\subsection{Organisation unicycle}
Le calcul du temps d'exécution en cycle d'horloge pour une organisation unicycle est aussi simple que de compter le nombre d'instructions exécuté. Aucune bulles ou vidanges est 
nécessaire. 

La boucle interne a un total de $15$ instructions qui une fois traduite donne un total de $23$. Ces 23 instructions sont exécuté $4$ fois. Au 5e appel, une seule instruction est
exécuté, soit la comparaison \verb|beq|, qui branche dans \verb|finBoucleInterne| qui est $2$ instructions. La formule pour la boucle interne est donc:
$$
T = 4\times(23)+(2+1)
$$

la boucle externe est aussi exécuté $4$ fois. Elle contient $2$ instructions puis une exécution complète de la boucle interne. À la 5e exécution, une seule instruction est exécuté
pour sortie de la boucle et appeller \verb|finBoucleExterne|. Cette dernière est $2$ instructions. La formule pour la boucle externe est donc la suivante:
$$
T = 4\times(2+4\times(23)+(2+1)) + (2+1)
$$
Reste juste le \verb|main| à inclure, qui est $2$ instructions. Le temps d'exécution en coup d'horloge est donc:
$$
T = 2+ 4\times(2+4\times(23)+(2+1)) + (2+1)
$$
$$
T = 4\times(4\times(23)+5) + 5
$$
$$
T = 4\times(97) + 5
$$
$$
T = 393
$$


%\todo{documenter sous forme algébrique l'organisation unicycle}
%\todo{Calculer en cycles d'horloge l'organisation unicycle}

\subsection{Organisation en pipeline}
%\todo{identifier et calcuer toutes les pénalités causées par deux organisation avec branchement au 4e étage en pipeline}
%\todo{identifier et calcuer toutes les pénalités causées par deux organisation avec branchement au 2e étage}

\subsection{Temps d'exécution}
%\todo{calculez le temps d'exécution en vous basant sur une vitesses d'opération de 25 ns pour l'organisation unicycle}
%\todo{calculez le temps d'exécution en vous basant sur une vitesses d'opération de 10 ns pour l'organisation pipeline}





\section{Performances SIMD}

\todo{identifiez les instructions qui seraient à convertir en SIMD}

\todo{calculez le nouveau temps d'exécution en cycles d'horloge, pour enfin le
comparer avec celui en unicycle}

\todo{dire si le gain de performance permet d'espérer d'atteindre les objectifs
de la problématique}

\section{Performances des mémoires sur processeur unicycle}

\section{Configuration des caches}

\section{Intégration}


\end{document}
